\chapter{开发过程记录}

开发过程全部记录于项目目录下doc/log.md。本章节选了部分内容。

\section{节选内容}

\subsection{11.29 下午}

让chatgpt生成了个wrapper封装了LibreHardwareMonitorLib,但是返回的数据没有CPU温度和功率,GitHub上查了查issue,说是因为管理员权限的问题,用管理员就能出电压了,但是也不出温度。

又查了下,有些数据被华硕主板设置为不可访问,我需要研究研究g-helper是怎么读到他们的。另外拯救者也有类似的设计,看来需要对一些牌子做额外适配。

不对啊,在LibreHardwareMonitor的UI里是能看到CPU温度和功率的,显然是封装有问题。让AI也没调好,看库的C\#代码也没看明白人家是怎么调用的,怪了。

github上找到一个人做了LibreHardwareMonitor的封装,但是他怎么也不写文档,没看明白怎么用。

不过风扇转速这样的东西确实是是被华硕藏起来的,必须得研究研究g-helper。

你们为什么不写文档啊???

\subsection{12.4 晚上}

HardwareInfoManager的constructor和update()里都需要处理硬件信息。同一个索引在这两个地方对应的名字是不一样的,假如手动写一一对应的代码没什么技术含量但维护负担太大。

所以我起初考虑写一个Map,然后用反射。但是反射遍历起来未免也太慢了,update()最好是x ms级的。

所以最后干脆用python写了个脚本(resources/sensor\_map.py)生成java代码。只需要维护脚本里那个表就行了。现在constructor和update()里巨长的if-else就是用它生成的,乐。

另外就是CPU、GPU这些类里成员变量很多,起初想用lombok简化Getter和Setter,但是不知道为什么过不了编译,干脆就用IDEA生成了一堆Getter和Setter,以后再折腾这个。

写了个test,现在的update()平均需要跑2ms,感觉还有优化空间(虽然完全够用了)。

\subsection{12.10 凌晨}

今天在写磁盘和注册表相关的东西。需要用到diskpart,是用process的BufferedWriter和BufferedReader交互的。这东西不太聪明,最开始的实现会卡住,认真看了这俩的文档发现我的用法有问题。后来明明已经consume了最开始的几行但是它仍旧会被读到。于是尝试了一些小妙招但是要么又卡住,要么返回空字符串,不会解决了。

想着一个char一个char读来debug,没想到这样就能得到正确结果了,莫名其妙解决问题了。

值得一提的是,这个过程中我一直在问sonnet 4.5,但是它给的方法并无什么用。让它debug十分钟不如我自己写几个println来debug两分钟。感觉是我没有掌握正确用法导致的,不然这也太菜了。

\subsection{12.16 上午}

今天想把battery的实现写出来,发现wrapper的实现很神经:当插电并在充电的时候,电池电流叫Charged Current,在放电的时候,叫Discharged Current,在插电,并且满电的时候叫Charged/discharged Current。就说是Hardware list是随着插电状态更新的,但是我只在程序最开始的时候读一次hardware list。这就导致我没法得知程序初始化后到底有没有在充电了。

花了不少时间研究能不能在不刷新hardware list的情况下知道到底读的是它们仨的谁,但是没想到。

不过我知道电脑充电的时候电流比充满时的大,因为充电时不仅有供电脑运行的电流,还有充电电流。又因为在插着电的时候电池是不放电的,所以此时要是有电流,那么就是充电电流,就说明电池没有充满。这样就得到一个判断方法:capacity下降就是放电,上升就是充电,current=0就是充满(因为充电、放电都体现在这个current上)。不过处于未知原因有时候current读不对,因此实际程序用rate(充放电功率)来代替了。

\subsection{12.19 凌晨}

回看日志,12.4那天晚上我写了一个生成代码用的python脚本。当时图省事直接用的元组列表,现在报应来了。

元组中有一项(假设是A)用来生成类似于cpu.A(lhmHelper.getValue(index[5]));的代码,但是我现在需要的是cpu.A(index[5]);,这就没办法了。所以我只好把A写成A(index[5]); //,这样生成的代码就是cpu.A(index[5]); //(lhmHelper.getValue(index[5]));,就能正常用了,跟SQL注入似的。
