\chapter{个人小结及建议}

很幸运的,我在高中时就接触了编程,第一门语言便是Java。一经入门我就喜欢上了这种创造的感觉。到了高二,我尝试用学到的知识开发一个背单词的项目\footnote{https://github.com/Potato-Yao/QiancizhanJava}。当时还没有AI,我不认识能请教的高手,也看不懂英文的论坛。每逢遇到技术问题,就只能在中文论坛里细细搜寻有用的信息,或者逐字逐句地翻看我的两本《Java核心技术》,一点一点反复尝试。每当代码终于跑通,我就能获得一种强烈且持久的快乐。转眼一看已经凌晨两点,第二天因为语文课上睡着被班主任制裁。

随着接触到的技术越多,我越认为自己会的太少,也就立下了想要探索更深入的计算机科学的志向。可以说Java是我的“母语”,为我打开了这伟大的计算机科学的大门。后来我在软件开发方面学习了C和Rust,还学习了设计模式之类的知识。上大学后我又自学了C++和Python,浅薄地了解了计算机组成原理和算法的知识,我还在网协学到了电脑的硬件组成和维修知识。

但是,在计算机学院的学习却与我的预想大相径庭。各种零碎的水课占据了我大量的时间,理论与实践割裂、教授内容平庸、安排不合理的某些专业课又极大阻碍了我的学习。上大学之后,我就越来越浮躁,反而没有写过太多代码,反而没有去深入学习计算机科学,不知道这样会持续到什么时候。

不过很幸运,这学期我选上了这门Java课。金老师不要求考勤,也没有布置任何琐碎的课业任务,给了我们极大的自由去进行适合自己节奏的学习,这也为我认真地开发项目提供了空间和契机。

开发这个项目的一个月,我经常想着“把这点改了就不写了”,但是一转眼就过去了一两个小时,又找到了那种强烈持久的愉悦。这真的让我非常高兴,我觉得我终于找到适合我的学习节奏了。

这学期电脑诊所招了新,为了培训和帮助新同学们干活,我每天晚上都在诊所,所以只参与了第一节和最后一节Java课。但是,每当我看到金老师在钉钉群里发的推文和PPT,我都能感受到一股强烈的热情,是“真的想要把知识给我们”。PPT大部分只是基础知识,但是包含了不少金老师自己的经验、想法,让我在阅读时总能获得新的知识。我觉得我课选的很幸运,也确实理解了学长学姐们强烈推荐的原因。

要说有什么不足的,就是没有抢到下学期的安卓开发。我甚至已经想好了大作业的选题,时间安排在上午我也保证可以全勤。但随机数真是不随人愿,深感遗憾。
